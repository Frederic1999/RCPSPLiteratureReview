\documentclass{paper}

\usepackage{xcolor}
\usepackage{amsmath}
\usepackage{listings}
%\usepackage{ulem}
\usepackage{csquotes}
\usepackage{siunitx}
\DeclareSIUnit\knot{kn}
\DeclareSIUnit\NM{NM}
\DeclareSIUnit\foot{ft}
\DeclareSIUnit\ftm{\foot\per\minute}
\usepackage[autolanguage]{numprint}
\definecolor{customcolor}{rgb}{1,0.5,0}
\usepackage{geometry}
\geometry{a4paper,left=20mm, top=25mm, right=20mm}
\usepackage{float}
\usepackage{amssymb}
\usepackage{cleveref}
\newtheorem{definition}{Definition}
\usepackage{amsmath}
\usepackage{graphicx}
\usepackage{mathtools}

\newcommand{\todo}[1]{{\color{red}{#1}}}
\def\intv#1[#2..#3]{\mathopen{#1[}#2 \mathrel{\mathinner {\ldotp \ldotp}}\nobreak #3\mathclose{#1]}}

\ExplSyntaxOn
\NewDocumentCommand \hms { o > { \SplitArgument { 2 } { ; } } m }
{
	\group_begin:
	\IfNoValueF {#1}
	{ \keys_set:nn { siunitx } {#1} }
	\siunitx_hms_output:nnn #2
	\group_end:
}
\cs_new_protected:Npn \siunitx_hms_output:nnn #1#2#3
{
	\IfNoValueF {#1}
	{
		\tl_if_blank:nF {#1}
		{
			\SI {#1} { \hour }
			\IfNoValueF {#2} { ~ }
		}
	}
	\IfNoValueF {#2}
	{
		\tl_if_blank:nF {#2}
		{
			\SI {#2} { \minute }
			\IfNoValueF {#3} { ~ }
		}
	}
	\IfNoValueF {#3}
	{ \tl_if_blank:nF {#3} { \SI {#3} { \second } } }
}
\ExplSyntaxOff

\DeclarePairedDelimiter\ceil{\lceil}{\rceil}


\begin{document}
	
	\title{Event-based MILP models for resource-constrained project scheduling problems}
	
	\maketitle
	
	\section*{Problem Description}
	
	\textbf{Resource-constrained project scheduling problem(RCPSP): NP-hard problem}
	\\A project involves activities, and resources (renewable or non-renewable), generally available in limited quantities. 
	The processing of an activity requires throughout its duration, one or more units of one or more resources. 
	The RCPSP deals with organizing in time the realization of activities, taking into account a number of precedence constraints, and constraints on the use and availability of the resources needed. 
	A schedule is a solution that describes resource allocation over time, and aims at satisfying one or more objectives.
	\begin{itemize}
		\item A combinatorial optimization problem defined by a tuple $(V,p,E,R,B,b)$
		\\$V$: activities, $p$:vector of durations, $E$: precedence relations,
		\\$R$: resources, $B$: vector of resource availabilities, $b$: matrix of demands
		\item  Activities constituting the project are identified by a set $\{0,\dots,n+1\}$.
		Activity $0$ represents by convention the start of the schedule, and activity $n + 1$ represents symmetrically the end of the schedule. 
		The set of non-dummy activities is identified by $\mathbf{A}=\{1,\dots,n\}$.
		\item $p\in N^{n+2}$, $p_i$ is the duration of activity $A_i$, with the special values: $p_0 = p_{n+1} = 0$
		\item $(i, j)\in E$ means that activity $i$ must precede activity $j$
		\item $B\in N^m$, $B_k$ is the availability of resource $k$, $m$ is the number of available resources
		\item b is an $(n+2)\times m$ integer matrix, $b_{ik}$ represents the amount of resource $k$ used per time period during the execution of activity $i$,
		$b_{0k} = 0$ , $b_{n+1,k} = 0$, for all $k \in R$.
		\item Schedule $S\in N^{n+2}$, $S_i$ represents the start time of activity $i$, $S_0=0$, $\mathbf{H}=\{1,\dots,T\}$ is the scheduling horizon, 
		and $T$ (the length of the scheduling horizon) is some upper bound for the makespan.
	\end{itemize} 
	
	\section{Basic Discrete-Time Formulation (DT)}
	   \subsection{Decision Variables}
	      \begin{itemize}
		     \item $x_{it}=1$ if activity $i$ starts at time $t$, $x_{it}=0$ otherwise
	      \end{itemize}
	   \subsection{Objective Function}
	      \begin{itemize}
		     \item $\min\sum _{t\in H}tx_{n+1,t}$ 
	      \end{itemize}
	   \subsection{Constraints}
	      \begin{itemize}
		     \item Precedence:
		    \begin{equation}
			 \forall (i,j) \in E, \quad \sum _{t\in H}tx_{jt}\geq \sum _{t\in H}tx_{it}+p_i
			\end{equation}
			\item Resource:
			\begin{equation}
			\forall t \in H, \forall k \in R, \quad \sum _{i=1}^{n}b_{ik}\sum _{\tau =t-p_i+1}^{t}x_{i\tau }\leq B_k
		    \end{equation}
			\item Non-preemptive:
			\begin{equation}
			\forall i \in A\cup \{0,n+1\}, \quad \sum _{t\in H}x_{it}=1
		    \end{equation}
		  \end{itemize}
	   \subsection{Problem Size}
	      \begin{itemize}
            \item Binary Variables: $(n+2)(T+1)$ 
			\item Constraints: $\left\lvert E \right\rvert + (T+1)m+n+2$ 
		  \end{itemize}
	
	\section{Disaggregated discrete-time formulation (DDT)}
	  \begin{itemize}
		\item The only diffference between DT and DDT is how they formulate the precedence constraints.
		\begin{equation}
			\forall t \in H, \forall (i,j) \in E, \quad \sum_{\tau =t}^{T} x_{i\tau}+\sum_{\tau =0}^{t+p_i-1} x_{j\tau}\leq 1
		\end{equation}
	  \end{itemize}
	  \begin{itemize}
		\item Problem Size
		\begin{itemize}
		  \item Binary Variables: $(n+2)(T+1)$ 
		  \item Constraints: $(m+\left\lvert E \right\rvert)(T+1)+n+2$ 
		\end{itemize}
	  \end{itemize}

	\section{Flow-based continuous-time formulation (FCT)}
	\subsection{Decision Variables}
	\begin{itemize}
	\item Starting-time continuous variables $S_i$, for each activity $i$
	\item Sequential variables $x_{ij}$ which are binary and indicate whether activity $i$ is processed before activity $j$
	\item Continuous flow variables $f_{ijk}$ to denote the quantity of resource $k$ that is transferred 
	from activity $i$ (at the end of its processing) to activity $j$ (at the start of its processing)
	\end{itemize}
	\subsection{Objective Function}
	\begin{itemize}
	  \item $\min S_{n+1}$
	\end{itemize}
	\subsection{Constraints}
	\begin{itemize}
		\item Precedence:
	   \begin{equation}
		\forall (i,j) \in (A\cup \{0,n+1\})^2, i\textless j, \quad x_{ij}+x{ji}\leq 1 
	   \end{equation}
	   \begin{equation}
		\forall (i,j,k) \in (A\cup \{0,n+1\})^3, \quad x_{ik}\geq x_{ij}+x_{jk}-1 
	   \end{equation}
	   \begin{equation}
		\forall (i,j) \in (A\cup \{0,n+1\})^2, \quad S_j-S_i\geq -M+(P_i+M)x_{ij} 
	   \end{equation}
	   \\Constraints (5) state that for two distinct activities, either $i$ precedes $j$, or $j$ precedes $i$, or $i$ and $j$ are processed in parallel. 
	   Constraints (6) express the transitivity of the precedence relations.
	   $M$ is some large enough constant. $M$ can be set to any valid upper bound of the makespan(e.g. $M=\sum_{i = 1}^{n}p_i$).
	   \item Resource:
	   \begin{equation}
	   \forall (i,j) \in (A\cup \{0\}\times A\cup \{n+1\}),\forall k \in R, \quad f_{ijk}\leq \min(b_{ik},b_{jk})x_{ij}
	   \end{equation}
	   \begin{equation}
         \forall j \in A\cup \{0,n+1\}, \forall k \in R, \quad \sum_{j\in A\cup \{0,n+1\}} f_{ijk}=\overline{b_{ik}} 
	   \end{equation}
	   \begin{equation}
		\forall i \in A\cup \{0,n+1\}, \forall k \in R, \quad \sum_{i\in A\cup \{0,n+1\}} f_{ijk}=\overline{b_{jk}} 
	   \end{equation}
	   \begin{equation}
		\forall k \in R, \quad f_{(n+1)0k}=B_k
	   \end{equation}
	   \\Constraint (8) indicates that if $i$ precedes $j$, the maximum flow sent from $i$ to $j$ is set to $\min \{b_i,b_j\}$ while if $i$ does not precede $j$ the flow must be zero. 
	   Constraints (9) and (10) are resource flow conservation constraints. 
	   Constraint (11) ensures the conservation of the flow.
	\end{itemize}
	
	\subsection{Problem Size}
	\begin{itemize}
		\item Binary Variables: $(n+2)^2$
		\item Continous Variables: $m(n+2)^2+n+2$ 
		\item Constraints: $n^3+(m+\frac{15}{2})n^2+(4m+\frac{35}{2})n+5m+13$ 
	  \end{itemize}
	\section{Start/End Event-based formulation (SEE)}
	An event occurs when an activity starts or ends.
	$\varepsilon = \{0,1,\dots,n\}$ is the index set of the events.

	\subsection{Decision Variables}
	\begin{itemize}
	\item $x_{ie} = 1$ if activity $i$ starts at event $e$.
	\item $y_{ie} = 1$ if activity $i$ ends at event $e$.
	\item Continous variable $t_e$ represents the date of event $e$
	\item Continous variable $r_ek$ represents the quantity of resource k required immediately after event $e$
	\end{itemize}

	\subsection{Objective Function}
	\begin{itemize}
	\item $\min t_n$	
	\end{itemize}
	
	\subsection{Constraints}
	
	\begin{itemize}
	\item Activity i starts at event $e$ and ends at event $f$, then $t_f \geq t_e+p_i$: 
	\begin{equation}
		\forall (e,f) \in \varepsilon ^2, f\textgreater e, \forall i \in A, \quad t_f \geq t_e + p_ix_{ie}-p_i(1-y_{if})
	\end{equation}
	\item A start event (respectively end event) has a single occurrence: 
	\begin{equation}
		\forall i \in A, \quad \sum_{e\in \varepsilon}x_{ie}=1, \sum_{e\in \varepsilon}y_{ie}=1
	\end{equation}
	\item The precedence relation between activities: If $i \textless j$ then $i$ ends at event $e$ or after, $j$ cannot start before event $e$.
	\begin{equation}
		\forall (i,j) \in E, \forall e \in \varepsilon ,\quad \sum_{e'= e}^{n}y_{ie'}+\sum_{e'= 0}^{e-1}x_{je'} \leq 1
	\end{equation}
	\item Resource conservation constraints that imply that for each resource $k$, its consumption immediately after event $e$ is equal to its consumption immediately
	after the previous event $e-1$, plus the consumption required by the activities that start at event $e$, minus the consumption required by the activities that end at event $e$.
	\begin{equation}
		\forall e \in \varepsilon, e\geq 1, \forall k\in R,\quad r_{ek}=r_{(e-1)k}+\sum_{i \in A}b_{ik}x_{ie}-\sum_{i \in A}b_{ik}y_{ie} 
	\end{equation}
	\item Limit the consumption of resources at each event to the availability of resources:
	\begin{equation}
		\forall e \in \varepsilon,\forall k \in R \quad r_{ek} \leq B_k
	\end{equation}
	\end{itemize}

	\subsection{Problem Size}
	\begin{itemize}
		\item Binary Variables: $2n(n+1)$
		\item Continous Variables: $n+1$ 
		\item Constraints: $\frac{1}{2}n^3+n^2+(3+\left\lvert E \right\rvert+m)n+\left\lvert E \right\rvert+m+1$ 
	  \end{itemize}
	
	\section{On/Off Event-based formulation (OOE)}
    \subsection{Decision Variables}
	\begin{itemize}
		\item $z_{ie} = 1$ if activity $i$ starts at event $e$ or if it still being processed immediately after event $e$.
		Thus, $z_{ie}$ remains equal to $1$ for the duration of the process activity $i$.
		\item Continous variable $t_e$ represents the date of event $e$
	\end{itemize}

	\subsection{Objective Function}
	\begin{itemize}
	\item $\min C_max$ , $C_{max}$ is the makespan.
	\end{itemize}

	\subsection{Constraints}
	\begin{itemize}
	  \item $C_{max} \geq t_e + p_i$ if $i$ is in process at event $e-1$ but not at event $e$
	\begin{equation}
	  \forall e \in \varepsilon,\forall i \in A \quad C_{max} \geq t_e + (z_{ie}-z_{i(e-1)})p_i
	\end{equation}
	\item Link the binary optimization variables $z_{ie}$ to the continuous optimization variables $t_e$, 
	and ensures that the processing time of an activity is equal to the processing time of this activity.
	\begin{equation}
		\forall (e,f,i) \in \varepsilon^2\times A,f \textgreater e\neq 0,\quad t_f\geq t_e+ ((z_{ie}-z_{i(e-1)})-(z_{if}-z_{i(f-1)})-1)p_i
	\end{equation}
	\item Ensure the adjacency of the events during which an activity being processed:
	\begin{equation}
		\forall e\neq 0 \in \varepsilon ,\quad \sum_{e'=0}^{e-1}z_{ie'}\geq e(1-(z_{ie}-z_{i(e-1)}))  
	\end{equation}	
	\begin{equation}
	    \forall e\neq 0 \in \varepsilon ,\quad \sum_{e'=e}^{n-1}z_{ie'}\geq e(1+(z_{ie}-z_{i(e-1)})) 
	\end{equation}	
	\item Each activity is processed at least once during the project:
	\begin{equation}
		\forall i \in A, \quad \sum_{e\in \varepsilon }z_{ie}\geq 1
	\end{equation}
	\item Precedence:
	\begin{equation}
		\forall e\in \varepsilon ,\forall (i,j)\in E,\quad z_{ie}+\sum_{e'=0}^{e}z_{je'}\leq 1+(1-z_{ie})e  
	\end{equation}
	\item Resource:
	\begin{equation}
		\forall e \in \varepsilon,\forall k \in R, \quad \sum_{i=0}^{n-1}b_{ik}z_{ie}\leq B_k 
	\end{equation}
	\end{itemize}

	\subsection{Problem Size}
	\begin{itemize}
		\item Binary Variables: $n^2$
		\item Continous Variables: $n+1$ 
		\item Constraints: $\frac{1}{2}n^3-\frac{1}{2}n^2+(3+\left\lvert E \right\rvert+m)n-2$ 
	\end{itemize}
	
\end{document}